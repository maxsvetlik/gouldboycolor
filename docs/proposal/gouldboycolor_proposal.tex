\documentclass{article}
\usepackage{setspace}
\usepackage{algorithm, algorithmic}
\newcommand\NoDo{\renewcommand\algorithmicdo{}}

\begin{document}
\begingroup
    \centering
    \LARGE Adv. Computer Arch Project Proposal \\
    \large Maxwell J Svetlik \\
    \large ms58323 \\
\endgroup

\section{Introduction} 
The Gameboy and later the Gameboy Color were among the first \textit{successful} portable handheld game systems. \\ 
These machines used fairly simple hardware to use gamedata from a cartridge and interact with the user. \\
In an effort to fully understand what it takes to emulate a CPU, and have a hands-on experience with concepts 
like DMA and I/O Memory Mapping, I will take on the task of emulating a Gameboy (Color) from scratch. \\
\\
This project will be publicly visible \footnote{https://github.com/maxsvetlik/gouldboycolor}
and version controlled through git.

\section{Administration}
    \subsection{Timeline}
        I see 5 main components to this project.\\
        \begin{enumerate}
            \item CPU Emulation
            \item Memory Management (Non I/O) 
            \item I/O \& Memory Mapping 
            \item Visualization 
            \item Sound 
        \end{enumerate}
        
        I will handle the first four, and if time permits, I'll add sound emulation.

        \subsection{Grading \& Checkpoint dates}
        \subsubsection{Dates}
        3/22 - CPU complete and Memory Mapping mostly complete. Other than commit history, some test cases, and project journal entires, I do not expect much 'visual' progress at this point.\\

        4/5 - Some visualization work. Bootup logo should be runable \footnote{https://www.youtube.com/watch?v=fZy08NsG2FM}, which requires pieces from 3 and 4 above. \\

        5/5 - Game loading, game visualization, and user interfacing in place. \\
        
        \subsubsection{Reports}
        For grading purposes, each checkpoint will come with a short report about the work differential from the previous checkpoint. These will describe in shallow detail what specific work has been done, pitfalls experienced, etc. Additionally, a commit history differential will be presented, as well as relevant project journal entries (more on this below). Grading can be given 
        according to adherence to the feature road-map above, or proper justification for missing it.
        
        \subsubsection{Note about the 5/5 checkpoint}
        I will not make guarantees about sound nor the smoothness of experience 
        in the final stage. I will consider a proper
        game load (loading memory from a cartridge), proper display of the image and 
        a successful user interaction (button press) without major complication or crashing 
        an overall success.

               
    \subsection{Resources \& Cheating}
        Given the popularity of the gameboy and the relative simplicity of the hardware, many emulators exist. 
        As a formal way to prove content originality, I will provide a project journal. 
        A project journal is something I keep for larger scale projects that detail the progression of my work. 
        More specifically, the entries usually mention what bugs or issues I'm facing and what I've 
        done to address those bugs or issues. This allows me to see what I've tried, what worked, and why 
        I made some of the design decisions I did. \\
        \\
        That said, these entries are not polished and will contain broken English, extended use of shorthand 
        and expletives. It should not be considered a formal report, but as a proof of work. Furthermore, I will provide a list of references (manuals, etc) used in the process.


\end{document}
